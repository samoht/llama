\documentclass[11pt]{book}
\usepackage[latin1]{inputenc}
\usepackage{alltt}
\usepackage{fullpage}
\usepackage{syntaxdef}
\usepackage{multind}
\usepackage{html}
\usepackage{caml-sl}
\usepackage{ocamldoc}
\newif\ifplaintext
\plaintextfalse

% Pour hevea
\newif\ifouthtml\outhtmltrue

\ifplaintext
\def\ttstretch{\tt}
\else
\def\ttstretch{\tt} %\spaceskip=5.77pt plus 1.83pt minus 1.22pt
% La fonte cmr10 a normalement des espaces de 5.25pt non extensibles.
% En 11 pt ca fait 5.77 pt. On lui ajoute la meme flexibilite que
% cmr10 agrandie a 11 pt.
\fi

\def\machine{\ttstretch}

%List of links with no space around items
\newstyle{.li-links}{margin:0ex 0ex;}
\newenvironment{links}
{\setenvclass{itemize}{ftoc2}\setenvclass{li-itemize}{li-links}\itemize}
{\enditemize}

\title{The Llama Light system}

\begin{document}

\tableofcontents

\part{An introduction to Llama}

\part{The Llama language}

\part{The Llama library}

\chapter{The general-purpose library} \label{c:genlib}

This chapter describes the functions provided by the Llama
general-purpose library. These modules can be used in standalone programs without
having to add any {\machine .cmo} file on the command line for the linking
phase. Similarly, in interactive use, these globals can be used in
toplevel phrases without having to load any {\machine .cmo} file in memory.

The functions provided by the following modules are generally
available, and can be referred to by their short names:

\begin{links}
\item \ahref{libref/builtin.html}{The built-in module: predefined types and exceptions}
\item \ahref{libref/Pervasives.html}{Module \texttt{Pervasives}: the initially opened module}
\end{links}

In contrast, the following modules are not automatically ``opened'' when a compilation
starts, or when the toplevel system is launched. Hence it is necessary
to use qualified identifiers to refer to the functions provided by these
modules, or to add {\machine open} directives.

\begin{links}
\item \ahref{libref/Arg.html}{Module \texttt{Arg}: parsing of command line arguments}
\item \ahref{libref/Array.html}{Module \texttt{Array}: array operations}
\item \ahref{libref/Buffer.html}{Module \texttt{Buffer}: extensible string buffers}
\item \ahref{libref/Callback.html}{Module \texttt{Callback}: registering Caml values with the C runtime}
\item \ahref{libref/Char.html}{Module \texttt{Char}: character operations}
\item \ahref{libref/Complex.html}{Module \texttt{Complex}: Complex numbers}
\item \ahref{libref/Digest.html}{Module \texttt{Digest}: MD5 message digest}
\item \ahref{libref/Filename.html}{Module \texttt{Filename}: operations on file names}
\item \ahref{libref/Format.html}{Module \texttt{Format}: pretty printing}
\item \ahref{libref/Gc.html}{Module \texttt{Gc}: memory management control and statistics; finalised values}
\item \ahref{libref/Genlex.html}{Module \texttt{Genlex}: a generic lexical analyzer}
\item \ahref{libref/Hashtbl.html}{Module \texttt{Hashtbl}: hash tables and hash functions}
\item \ahref{libref/Int32.html}{Module \texttt{Int32}: 32-bit integers}
\item \ahref{libref/Int64.html}{Module \texttt{Int64}: 64-bit integers}
\item \ahref{libref/Lexing.html}{Module \texttt{Lexing}: the run-time library for lexers generated by \texttt{llamalex}}
\item \ahref{libref/List.html}{Module \texttt{List}: list operations}
\item \ahref{libref/Map.html}{Module \texttt{Map}: association tables over ordered types}
\item \ahref{libref/Marshal.html}{Module \texttt{Marshal}: marshaling of data structures}
\item \ahref{libref/Nativeint.html}{Module \texttt{Nativeint}: processor-native integers}
\item \ahref{libref/Parsing.html}{Module \texttt{Parsing}: the run-time library for parsers generated by \texttt{llamayacc}}
\item \ahref{libref/Printexc.html}{Module \texttt{Printexc}: facilities for printing exceptions}
\item \ahref{libref/Printf.html}{Module \texttt{Printf}: formatting printing functions}
\item \ahref{libref/Queue.html}{Module \texttt{Queue}: first-in first-out queues}
\item \ahref{libref/Random.html}{Module \texttt{Random}: pseudo-random number generator (PRNG)}
\item \ahref{libref/Scanf.html}{Module \texttt{Scanf}: formatted input functions}
\item \ahref{libref/Set.html}{Module \texttt{Set}: sets over ordered types}
\item \ahref{libref/Sort.html}{Module \texttt{Sort}: sorting and merging lists}
\item \ahref{libref/Stack.html}{Module \texttt{Stack}: last-in first-out stacks}
\item \ahref{libref/Stream.html}{Module \texttt{Stream}: streams and parsers}
\item \ahref{libref/String.html}{Module \texttt{String}: string operations}
\item \ahref{libref/Sys.html}{Module \texttt{Sys}: system interface}
\item \ahref{libref/Weak.html}{Module \texttt{Weak}: arrays of weak pointers}
\end{links}
\else
\fi

\chapter{The metalinguistic library}

\ifouthtml
\begin{links}
\item \ahref{libref/Location.html}{Module \texttt{Location}: source code locations}
\end{links}
\else
\input{Location.tex}
\fi

\end{document}
