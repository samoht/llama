\documentclass[11pt]{article}
\usepackage[latin1]{inputenc}
\usepackage{alltt}
\usepackage{fullpage}
\usepackage{syntaxdef}
\usepackage{multind}
\usepackage{html}
\usepackage{caml-sl}
\usepackage{ocamldoc}
\usepackage{color}
\newif\ifplaintext
\plaintextfalse

%List of links with no space around items
\newstyle{.li-links}{margin:0ex 0ex;}
\newenvironment{links}
{\setenvclass{itemize}{ftoc2}\setenvclass{li-itemize}{li-links}\itemize}
{\enditemize}

\newstyle{.title}{padding:1ex;background:\#00B200;}
\newstyle{.titlemain}{padding:1ex;background:\#00B200;}
\newstyle{.titlerest}{padding:1ex;background:\#00B200;}
\newstyle{.part}{padding:1ex;background:\#00CC00;}
\newstyle{.section}{padding:.5ex;background:\#2DE52D;}
\newstyle{.subsection}{padding:0.3ex;background:\#66FF66;}
\newstyle{.subsubsection}{padding:0.5ex;background:\#99FF99;}
\newstyle{.toc}{list-style:disc;}

\def\isalso{ & += & \spacefalse }

\begin{document}

\begin{center}
\Huge  The Llama system \\
\large Jeremy Bem \\
       drawing on the work of Xavier Leroy and many others \\
\end{center}

\tableofcontents

\section{Introduction}

Currently this manual assumes a familiarity with
\ahref{http://caml.inria.fr/}{Objective Caml}, which is Llama's
immediate ancestor. Readers lacking this prerequisite (or at least
familiarity with a similar language such as Standard ML or Haskell)
are advised to first study that system, working through the tutorial
and such. It will be time well spent.

\section{Llama Light}

\subsection{The language}

\subsection{The library}

\begin{links}
\item \ahref{libref/builtin.html}{The built-in module: predefined
  types and exceptions}
\item \ahref{libref/Pervasives.html}{Module \texttt{Pervasives}: the initially opened module}
\end{links}

\begin{links}
\item \ahref{libref/Arg.html}{Module \texttt{Arg}: parsing of command line arguments}
\item \ahref{libref/Arith_status.html}{Module \texttt{Arith\_status}: flags that control rational arithmetic}
\item \ahref{libref/Array.html}{Module \texttt{Array}: array operations}
\item \ahref{libref/Big_int.html}{Module \texttt{Big\_int}: operations on arbitrary-precision integers}
\item \ahref{libref/Buffer.html}{Module \texttt{Buffer}: extensible string buffers}
\item \ahref{libref/Callback.html}{Module \texttt{Callback}: registering Caml values with the C runtime}
\item \ahref{libref/Char.html}{Module \texttt{Char}: character operations}
\item \ahref{libref/Complex.html}{Module \texttt{Complex}: Complex numbers}
\item \ahref{libref/Digest.html}{Module \texttt{Digest}: MD5 message digest}
\item \ahref{libref/Filename.html}{Module \texttt{Filename}: operations on file names}
\item \ahref{libref/Format.html}{Module \texttt{Format}: pretty printing}
\item \ahref{libref/Gc.html}{Module \texttt{Gc}: memory management control and statistics; finalised values}
\item \ahref{libref/Genlex.html}{Module \texttt{Genlex}: a generic lexical analyzer}
\item \ahref{libref/Hashtbl.html}{Module \texttt{Hashtbl}: hash tables and hash functions}
\item \ahref{libref/Int32.html}{Module \texttt{Int32}: 32-bit integers}
\item \ahref{libref/Int64.html}{Module \texttt{Int64}: 64-bit integers}
\item \ahref{libref/Lexing.html}{Module \texttt{Lexing}: the run-time library for lexers generated by \texttt{llamalex}}
\item \ahref{libref/List.html}{Module \texttt{List}: list operations}
\item \ahref{libref/Map.html}{Module \texttt{Map}: association tables over ordered types}
\item \ahref{libref/Marshal.html}{Module \texttt{Marshal}: marshaling of data structures}
\item \ahref{libref/Nativeint.html}{Module \texttt{Nativeint}: processor-native integers}
\item \ahref{libref/Num.html}{Module \texttt{Num}: operations on arbitrary-precision numbers}
\item \ahref{libref/Parsing.html}{Module \texttt{Parsing}: the run-time library for parsers generated by \texttt{llamayacc}}
\item \ahref{libref/Printexc.html}{Module \texttt{Printexc}: facilities for printing exceptions}
\item \ahref{libref/Printf.html}{Module \texttt{Printf}: formatting printing functions}
\item \ahref{libref/Queue.html}{Module \texttt{Queue}: first-in first-out queues}
\item \ahref{libref/Random.html}{Module \texttt{Random}: pseudo-random number generator (PRNG)}
\item \ahref{libref/Scanf.html}{Module \texttt{Scanf}: formatted input functions}
\item \ahref{libref/Set.html}{Module \texttt{Set}: sets over ordered types}
\item \ahref{libref/Sort.html}{Module \texttt{Sort}: sorting and merging lists}
\item \ahref{libref/Stack.html}{Module \texttt{Stack}: last-in first-out stacks}
\item \ahref{libref/Stream.html}{Module \texttt{Stream}: streams and parsers}
\item \ahref{libref/Str.html}{Module \texttt{Str}: regular expressions and string processing}
\item \ahref{libref/String.html}{Module \texttt{String}: string operations}
\item \ahref{libref/Sys.html}{Module \texttt{Sys}: system interface}
\item \ahref{libref/Weak.html}{Module \texttt{Weak}: arrays of weak pointers}
\end{links}

\begin{links}
\item \ahref{libref/Location.html}{Module \texttt{Location}: source code locations}
\end{links}

\subsection{The tools}

\section{Deductive Llama}

\begin{syntax}
\nonterm{expr}\isalso{}\nonterm{binder-op} \nonterm{pattern} \token{,} \nonterm{expr}
\end{syntax} \begin{syntax}
\nonterm{binder-op}\is{}\token{forall}
 \alt{}\token{exists}
 \alt{}\token{exists\_unique}
\end{syntax}

\section{Inductive Llama}

\end{document}
